\documentclass[a4paper,11pt]{article}
\usepackage[left=1.8cm, right=1.8cm, top=2.5cm, bottom=2cm]{geometry}
\usepackage{xeCJK}
\usepackage{indentfirst}
\usepackage{tabularx}
\usepackage{graphicx}
\usepackage{amsmath}
\usepackage{listings}
\usepackage{verbatim}
\usepackage{fancyhdr}
% \usepackage[compact]{titlesec}
\usepackage[usenames,dvipsnames]{xcolor}

\setCJKmainfont{NotoSansCJKtc-Thin}
\setmonofont{Consolas}

\definecolor{CodeGreen}{rgb}{0,0.6,0}
\definecolor{CodeGray}{rgb}{0.5,0.5,0.5}
\definecolor{CodeMauve}{rgb}{0.58,0,0.82}
\lstset{
    basicstyle = \ttfamily\footnotesize, 
    breakatwhitespace = false,
    breaklines = true,         
    commentstyle = \color{CodeGreen}\bfseries,
    extendedchars = false,
    keepspaces=true,
    keywordstyle=\color{blue}\bfseries, % keyword style
    language = C++,                     % the language of code
    otherkeywords={string},
    numbers=none,
    numbersep=5pt,
    numberstyle=\tiny\color{CodeGray},
    rulecolor=\color{black},
    showspaces=false,
    showstringspaces=false,
    showtabs=false,
    stepnumber=1,       
    stringstyle=\color{CodeMauve},        % string literal style
    tabsize=2,
}
% from https://blog.csdn.net/RobertChenGuangzhi/article/details/45126785

\title{Machine Learning 2020 - Homework 3 Report}
\author{學號:b08902100, 系級:資工一, 姓名:江昱勳}
\date{}

\begin{document}
\pagestyle{fancy}
\fancyhead[L]{Machine Learning 2020 - Homework 3}
\fancyhead[R]{Author: b08902100 江昱勳}

\maketitle

% \verbatiminput{HW2_S.txt}
% \lstinputlisting{HW2.cpp}

\begin{enumerate}

\item 請說明你實作的 CNN 模型,其模型架構、訓練參數量和準確率為何?

模型結構如下:
\begin{lstlisting}[language={}]
Layer (type)             Output Shape          Param #
=======================================================
     Conv2d        [-1, 128, 128, 128]           3,584
BatchNorm2d        [-1, 128, 128, 128]             256
       ReLU        [-1, 128, 128, 128]               0
  MaxPool2d          [-1, 128, 64, 64]               0
  Dropout2d          [-1, 128, 64, 64]               0
     Conv2d          [-1, 128, 64, 64]         147,584
BatchNorm2d          [-1, 128, 64, 64]             256
       ReLU          [-1, 128, 64, 64]               0
  MaxPool2d          [-1, 128, 32, 32]               0
     Conv2d          [-1, 256, 32, 32]         295,168
BatchNorm2d          [-1, 256, 32, 32]             512
      PReLU          [-1, 256, 32, 32]               1
  MaxPool2d          [-1, 256, 16, 16]               0
  Dropout2d          [-1, 256, 16, 16]               0
     Conv2d          [-1, 512, 16, 16]       1,180,160
BatchNorm2d          [-1, 512, 16, 16]           1,024
       ReLU          [-1, 512, 16, 16]               0
  MaxPool2d            [-1, 512, 8, 8]               0
     Conv2d            [-1, 512, 8, 8]       2,359,808
BatchNorm2d            [-1, 512, 8, 8]           1,024
       ReLU            [-1, 512, 8, 8]               0
  MaxPool2d            [-1, 512, 4, 4]               0
     Linear                  [-1, 512]       4,194,816
       ReLU                  [-1, 512]               0
     Linear                  [-1, 256]         131,328
       ReLU                  [-1, 256]               0
     Linear                  [-1, 128]          32,896
    Dropout                  [-1, 128]               0
       ReLU                  [-1, 128]               0
     Linear                  [-1, 100]          12,900
       ReLU                  [-1, 100]               0
     Linear                   [-1, 30]           3,030
      PReLU                   [-1, 30]               1
     Linear                   [-1, 11]             341
================================================================
Total params: 8,364,689
\end{lstlisting}
訓練400個epoch後在Kaggle上的public score為0.84997,在訓練過程中,我將原先的資料複製一份並且分別通過兩種不同的變換如下,而最後我也有normalize過。

\begin{lstlisting}[language=Python]
train_transform1 = transforms.Compose([
  transforms.ToPILImage(),
  transforms.RandomChoice([
    transforms.RandomVerticalFlip(),
    transforms.RandomHorizontalFlip(),
    transforms.RandomPerspective()
  ]),
  transforms.RandomChoice([
    transforms.RandomAffine(10),
    transforms.RandomRotation(40)
  ]),
  transforms.ColorJitter(),
  transforms.ToTensor(),
  transforms.Normalize(
    [77.89311144813877 / 255, 102.3587941606983 / 255, 126.59376063616554 / 255],
    [72.80305392379675 / 255, 75.35438507973123 / 255, 79.31408066842762 / 255]
  )
])
train_transform2 = transforms.Compose([
  transforms.ToPILImage(),
  transforms.RandomOrder([
    transforms.RandomChoice([
      transforms.RandomHorizontalFlip(),
      transforms.RandomPerspective()
    ]),
    transforms.RandomAffine(30),
    transforms.RandomResizedCrop((128, 128), scale=(0.5, 1.0)),
  ]),
  transforms.RandomChoice([
    transforms.ColorJitter(),
    transforms.RandomGrayscale(),
  ]),
  transforms.ToTensor(),
  transforms.RandomErasing(0.2),
  transforms.Normalize(
    [77.89311144813877 / 255, 102.3587941606983 / 255, 126.59376063616554 / 255],
    [72.80305392379675 / 255, 75.35438507973123 / 255, 79.31408066842762 / 255]
  )
])
\end{lstlisting}

\item 請實作與第一題接近的參數量,但 CNN 深度(CNN 層數)減半的模型,並說明其模型架構、訓練參數量和準確率為何?

模型如下:
\begin{lstlisting}
Layer (type)               Output Shape         Param #
=======================================================
     Conv2d        [-1, 128, 128, 128]           3,584
BatchNorm2d        [-1, 128, 128, 128]             256
       ReLU        [-1, 128, 128, 128]               0
  MaxPool2d          [-1, 128, 64, 64]               0
  Dropout2d          [-1, 128, 64, 64]               0
     Conv2d          [-1, 256, 64, 64]         295,168
BatchNorm2d          [-1, 256, 64, 64]             512
       ReLU          [-1, 256, 64, 64]               0
  MaxPool2d          [-1, 256, 16, 16]               0
     Conv2d          [-1, 512, 16, 16]       1,180,160
BatchNorm2d          [-1, 512, 16, 16]           1,024
       ReLU          [-1, 512, 16, 16]               0
  MaxPool2d            [-1, 512, 4, 4]               0
     Linear                  [-1, 800]       6,554,400
    Dropout                  [-1, 800]               0
       ReLU                  [-1, 800]               0
     Linear                  [-1, 400]         320,400
      PReLU                  [-1, 400]               1
     Linear                  [-1, 256]         102,656
       ReLU                  [-1, 256]               0
     Linear                  [-1, 128]          32,896
    Dropout                  [-1, 128]               0
       ReLU                  [-1, 128]               0
     Linear                   [-1, 50]           6,450
       ReLU                   [-1, 50]               0
     Linear                   [-1, 30]           1,530
      PReLU                   [-1, 30]               1
     Linear                   [-1, 11]             341
========================================================
Total params: 8,499,379
\end{lstlisting}

在kaggle上的分數為0.84518,跟沒有砍一半的差不多,所以我猜測其實第一題的模型還有一些可以調整的地方。

\item 請實作與第一題接近的參數量,簡單的 DNN 模型,同時也說明其模型架構、訓練參數和準確率為何?

模型如下:
\begin{lstlisting}[language={}]
Layer (type)        Output Shape         Param #
==================================================
     Linear            [-1, 180]       8,847,540
       ReLU            [-1, 180]               0
     Linear            [-1, 128]          23,168
    Dropout            [-1, 128]               0
       ReLU            [-1, 128]               0
     Linear             [-1, 64]           8,256
       ReLU             [-1, 64]               0
     Linear             [-1, 30]           1,950
      PReLU             [-1, 30]               1
     Linear             [-1, 11]             341
=================================================
Total params: 8,881,256
\end{lstlisting}

在kaggle上的分數為0.45845。

\item 請說明由 1 ~ 3 題的實驗中你觀察到了什麼?

從第三題我們可以知道在同樣的參數量下,CNN對於圖片能提供更好的解析能力,而減少CNN的層數也會讓模型對於圖像的分辨能力降低,因此可以知道CNN對於影像辨識能有很好的幫助。

\item 請嘗試 data normalization 及 data augmentation,說明實作方法並且說明實行前後對準確率有什麼樣的影響?

第一題的模型有使用data normalization 與 data augmentation,而若是沒有data augmentation的話在kaggle上的accuracy降到0.64734,可以知道我們的模型很需要較多的資料,而取消normalization後,kaggle上的分數也降到0.84339了,可以知道我們的模型也對於資料的品質有一定的要求,只是相比資料量,沒有那麼需要而已。

\item 請說明你實作的 best model,其模型架構、訓練參數量和準確率為何?

參考VGG的架構後,將model分為捲積層以及全連接層,全連接層的部分由三個線性層組成,中間會使用batch norm並且使用ReLU作為活化函數;捲積層則是都用大小為三的kernel,除了第一層以外,接下來每一層都是由兩層CNN組成,接著再接一層大小為2的max pool。總訓練參數為 22269707 ,而在Kaggle上的public score為0.88404。

\item 觀察答錯的圖片中,哪些 class 彼此間容易用混?[繪出 confusion matrix 分析]

\begin{figure}[ht]
    \center
    \includegraphics[width=.8\textwidth]{confusion.pdf}
    \caption{confusion matrix}
    \label{fig:conf}
\end{figure}

從圖\ref{fig:conf}中可以看到,這個model最容易將3跟0搞混,1跟2之間也很容易被搞混。

\end{enumerate}

\end{document}